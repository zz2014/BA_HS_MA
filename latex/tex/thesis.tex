% Preambel mit Einstellungen importieren
\input{preambel}

% Dokumenteninfos importieren
% -------------------------------------------------------
% Daten für die Arbeit
% Wenn hier alles korrekt eingetragen wurde, wird das Titelblatt
% automatisch generiert. D.h. die Datei titelblatt.tex muss nicht mehr
% angepasst werden.

\newcommand{\hsmasprache}{en} % de oder en für Deutsch oder Englisch

% Titel der Arbeit auf Deutsch
%\newcommand{\hsmatitelde}{Einsatz eines Flux-Kompensators für Zeitreisen mit einer maximalen Höchstgeschwindigkeit von WARP~7}

% Titel der Arbeit auf Englisch
\newcommand{\hsmatitelen}{Performance Comparison between JAVA and Node.js in terms of Scaling in Cloud}

% Weitere Informationen zur Arbeit
\newcommand{\hsmaort}{Mannheim}    % Ort
\newcommand{\hsmaautorvname}{Zheng} % Vorname(n)
\newcommand{\hsmaautornname}{Zeng} % Nachname(n)
\newcommand{\hsmadatum}{11.11.2015} % Datum der Abgabe
\newcommand{\hsmajahr}{2015} % Jahr der Abgabe
\newcommand{\hsmafirma}{Paukenschlag GmbH, Mannheim} % Firma bei der die Arbeit durchgeführt wurde
\newcommand{\hsmabetreuer}{Prof. Peter Knauber, Hochschule Mannheim} % Betreuer an der Hochschule
\newcommand{\hsmazweitkorrektor}{Jens Keller,  SAP SE} % Betreuer im Unternehmen oder Zweitkorrektor
\newcommand{\hsmafakultaet}{I} % I für Informatik
\newcommand{\hsmastudiengang}{IB} % IB IMB UIB IM MTB

% Zustimmung zur Veröffentlichung
\setboolean{hsmapublizieren}{true}   % Einer Veröffentlichung wird zugestimmt
\setboolean{hsmasperrvermerk}{false} % Die Arbeit hat keinen Sperrvermerk

% -------------------------------------------------------
% Abstract

% Kurze (maximal halbseitige) Beschreibung, worum es in der Arbeit geht auf Deutsch
\newcommand{\hsmaabstractde}{}

% Kurze (maximal halbseitige) Beschreibung, worum es in der Arbeit geht auf Englisch

\newcommand{\hsmaabstracten}{}





\begin{document}
\frontmatter

% Römische Ziffern für die "Front-Matter"
\setcounter{page}{0}
\changefont{ptm}{m}{n}  % Times New Roman für den Fließtext
\renewcommand{\rmdefault}{ptm}

% Titelblatt
\input{titelblatt}

% Inhaltsverzeichnis erzeugen
\cleardoublepage
\pdfbookmark{\contentsname}{Contents}
\tableofcontents

% Korrigiert Nummerierung bei mehrseitigem Inhaltsverzeichnis
\cleardoublepage
\newcounter{frontmatterpage}
\setcounter{frontmatterpage}{\value{page}}

% Arabische Zahlen für den Hauptteil
\mainmatter

% Den Hauptteil mit vergrößertem Zeilenabstand setzen
\onehalfspacing

% ------------------------------------------------------------------
% Hauptteil der Arbeit
% Die Arbeit besteht aus Kapiteln (chapter)
\chapter{Introduction}
The promise of cloud has ushered the software engineering into a new era. Technology is to be used anytime and anywhere to untangle issues, provide solutions, bring people together and change their ways of living. This new era of net-centric web-based applications redefines how software is delivered to a customer and how the customer uses the delivered software. The digitization of economy impacts everyone. New businesses and leaders are emerging from nowhere. Companies appear and disappear much faster than ever before. 

SAP, a company which has been renown for its on premise ERP solutions, also faces the pressure from customers to reduce \ac{TCO} and increase agility. It has since long announced its cloud strategy to help customer master the digital economy. The reach of SAP systems is to be extended through cloud based access, ultimately reaching everybody everywhere in entirely new ways. Cloud computing allows on demand software provisioning with Zero-Installation and automatic configuration at low cost and immediate access in ultra-scalable data centers, which leads to the next generation networked solutions. In the process of leverage cloud infrastructure to increase the business agility and to lower the TCO of customers, ultimately new types of applications are enabled.

Cloud solutions come with new capabilities and technical challenges for cloud software development. Different cloud software development platforms and frameworks applied with different programming languages are used inside SAP. Java and Node.JS are the most prominent ones. There have been on-going discussions about how to  take a choice between them for SAP applications considering the SAP environment.\\
Among all the major advantages brought by the cloud paradigm, scalability is the one that makes cloud computing different to an "advanced outsourcing" solution. It’s no secret: high-performance website and web applications drive more traffic, engagement, and increase brand loyalty. In this age where customers are won and lost in a second, continual evaluation and optimization of web properties is essential.  Performance affects customer satisfaction. Slow downs and/or downtime can cost companies thousands of dollars.\\
On account of all the factors named above,  the thesis aims to compare the two server-side technologies: Java and Node.js, in terms of their respective performance and scalibility in a PaaS cloud offering: Cloud Foundry.






 % Externe Datei einbinden\\
\chapter{Related Work}
Extensive research has been done in the field of comparative studies of programming language. There have been lots of studies evaluating and analyzing Web server performance. Lei, Ma and Tan has studied and compared the performance of Node.js, Python-Web and PHP using benchmark tests and scenario tests\citep{node-related}. The result shows that Node.js is an idea fit for I/O intensive websites. A comparison betwwen Java and PHP is conducted by Neves, Paiva and Duraes in terms of performance and security\citep{java-related}. It is concluded that Java is more scalable and secure than PHP. \\
Although the comparison between Java and Node.js are quite rare among published paper and technological reports, a quite thorough comparison for almost all the web technologies are done by the TechEmpower \cite{Benchmark}. The tests are conducted in physical and cloud environment. From the results, one can see Java exceeds the Node.js in almost all aspects of performance. \\
In comparison with all the work mentioned above, this thesis has regulated the tests in a PaaS cloud offering. In addition to the performance of single application, the scalability is also identified. The focus of thesis is set on single transaction which is a main SAP business scenario.  % Externe Datei einbindens
\chapter{Experimental Environments}
\section{Server running environment -  Cloud Foundry}
\subsection{Cloud Foundry at SAP}
\subsection{PaaS - Service plans, Backing services and more}
\subsection{Limitations}
\section{ Client running environments}
\subsection{Influential factors for measuring}
\subsection{Client in local machine}
\subsection{Client in Cloud Foundry}
\subsection{Client in Monsoon}


 % Externe Datei einbinden
\chapter{Testing Tool}
\section{Load generator - a self implemented scalable client}
\todo[inline]{Write about load generator as simple node application}
\subsection{Design and function of load generator}
\label{load generator}

\subsection{Limitations}
\todo[inline]{Write about limitation of db for load generating} % Externe Datei einbinden
\chapter{Measuring Tool}
\section{Response time}
  \todo[inline]{Write why care about response time}
\subsection{Recording with Load Generator and its limitations}
  \todo[inline]{refer to previous chapter}
\subsection{Retrieving data from ELK and its limitations}
  \todo[inline]{Write ELK node application and log quota}
\section{CPU and Memory consumption}
  \todo[inline]{Write why care about cpu and memory}
\subsection{Cloud Foundry CLI and its limitations}
  \todo[inline]{Write: cf statistics}
\subsection{A self-implemented  Ruby application and its limitations}
  \todo[inline]{Do I need it?}


 % Externe Datei einbinden
\chapter{Implementation of Applications}
Store the current order of product and its meta information, find which shelf stores the current requested product, assign an idling logistic unit to pick up the itinerary ... I/O intensive operations make up the majority of SAP business scenarios. 
 In the load test conducted in this paper, applications are built to realize such a scenario: advertisements are published in a bulletin board and clients can browse through the items. 

The PostgreSQL backing service from Cloud Foundry is used as the database. As the goal is to test how the application handles large amount of concurrency instead of the database efficiency, data will not be queried in high quantity or by complicated SQL actions in the test. 

\section{Implementation Configuration}
To bring Java and Node.js to a comparable level, the applications are implemented with minimum use of frameworks so that the overhead or any other influential factors on performance can be first taken off the table.

 \begin{figure}[h]
 	\centering
 	\includegraphics[width=12cm]{implementation}
 	\caption{Implementation configuration}
 	\label{implementation}
 \end{figure}

As table \ref{implementation} shows,  the implementation of both applications utilize no REST or any other kind of web framework. No ORM is applied. Response is sent as plain text to the client to avoid overhead brought by using framework to do JSON serialization/deserialization. \\
Java application uses embedded Jetty server to handle plain HTTP requests because its lightweight. (* There is an even simpler built-in HTTP server from Oracle JRE. However, the build pack used in Cloud Foundry is OpenJDK JRE.)  Node.js application uses its embedded web platform. The connection with database is plain JDBC for Java while Node.js uses a popular PostgreSQL library: PG. Since the data structure is intentionally kept simple: only one table and with no complex data types, the application without ORM doesn't bring about a lot of boiler plate code.  \\

\section{Optimize the Java implementation}
All possible attempts are made to bring about every potential performance of the application. Since there is no complex logic, the focus of optimization lays on the interaction between applications and database. In case of Java implementations, to set a optimal thread pool configuration is also investigated. \\
The first checkpoint is database connection which greatly affect performance since it is the most expensive operation in the application without a complicated computing logic. Opening a connection and closing it with every request would gigantically slow down the application. Therefore connection pool is a key component in the implementation. It turns out there are quite a few libraries which handles connection pooling. In the thesis,  three different libraries are tried out. \textit{PGPoolingDataSource}  \citep{pgpool}   comes with default PostgreSQL JDBC driver. \textit{commons-dbcp2} \citep{dbcp} is from Apache Software Foundation. \textit{HikariCP} is a "zero-overhead" production ready connection pool. It turns out \textit{HikariCP} \citep{hikari} has outperformed the other two. \\
The next thing is to find an ideal configuration for the connection pool size. In an article from Brett Wooldridge \citep{poolsize}, it is pointed out larger connection pool size configuration doesn't necessarily lead to a better performance. Single core can only execute one thread at a time; then the OS switches contexts and that core executes code for another thread, and so on. Given a single CPU resource, executing A and B sequentially will always be faster than executing A and B "simultaneously" through time-slicing. Theoretically, the database will be slowed down the moment connection pool size exceeds the total core number. However, there are a few other factors at play. For example, databases typically store data on a disk, which traditionally is comprised of spinning plates of metal with read/write heads mounted on a stepper-motor driven arm. So there is a time cost for disk "I/O wait". During this time the OS could put that CPU resource to better use by executing some more code for another thread. Because threads become blocked on I/O,  more work can be done by having a number of connections/threads that is greater than the number of physical computing cores.
\begin{figure}[h]
	\centering
	\includegraphics[width=12cm]{pool_size_con_user}
	\caption{Database connection pool size comparison}
	\label{pool-comparison}
\end{figure}
In the case of thesis, database is running in Diego cell which contains 4 CPUs. In order to verify and find the best fit for the load the thesis intend to generate, an experiment is conducted with connection pool size of 30, 60, and 90.  Figure \ref{pool-comparison} shows a slight difference can be deducted that the pool size of 90 reaches the upper limit of total transaction slightly earlier than the other configurations. A little bit better is the performance from a connection pool size of 60 than that of 30. Then we compared the end-to-end response time with the data base response time.  As figure \ref{ete-vs-db} shows, when the response time peaks, the database response time stays stable and can hardly be considered responsible for the peak. Since the database pool size doesn't pose a conspicuous difference, we can draw the conclusion that the database is working at a high speed with no noticeable delay. The pool size setting is not a deciding bottleneck at all. It could be because the database is a docker container version and has no dedicated virtual machine. Thus it lies likely quite near the application and has little network overhead. 

\begin{figure}[h]
	\centering
	\includegraphics[width=12cm]{ete-vs-db}
	\caption{Comparing end-to-end response time with database response time}
	\label{ete-vs-db}
\end{figure}

Thread pool size is also scrutinized for Java application and following the recommendation made by Jetty \citep{threadpool}, the size is set from 10 to 400. \\


\section{Optimize the Node.js implementation}
Node.js application basically faces the same configuration of connection pool in database. In the thesis, also three different libraries are tried out. Unlike Java libraries, there is some overlapping with regard to the Node.js libraries. In npm, one can find a number of PostgreSQL drivers. However, majority of them are built on the basis of one library: "node-postgres/pg" \citep{node-pg}. They are either wrappers or additional implementation with "promise" or "async/wait". For example, a great difference can not be derived from using of "node-postgres/pg" and "pg-promise" in the scenario in this thesis. The comprehensive research on framework benchmarking \citep{Benchmark} uses "Sequelize", which is also tried out in the thesis. However, it yields even a worse performance result because the object mapping costs indisputably computing time. In the end, the suggestion from "node-postgres/pg" writer is adopted to use "pg-native" which can boost a 20-30\% increase in parsing speed.\\




 % Externe Datei einbinden
\chapter{Test Strategy, Results and Analysis}
\section{Test strategy}
At the beginning of the testing phase of the thesis, both applications are given the same amount of memory to ensure they have the same CPU share. It turns out, Java has nearly twice much throughput as the Node.js application. It seems one is comparing something which is totally not comparable at all. It is also quite far away from expectation. Things are tried and investigations are made to identify the cause. \\
The monitoring on CPU usage shows that the Java application can utilize up to 250\%,  in contrast which the Node.js application never quite exceeds the limit of 100\%. This has everything to do with the single-threaded principle of Node.js applications. While Java application is through adding CPU and memory vertical scalable, Node.js is single-threaded and scales by creating multiple-node processes. It is only fair to test both application when they utilize the same amount of CPUs. However, the Cloud Foundry specific way of designating computing resources ensures there is no neat cut of a piece of CPU unless one is completely alone in the land scape. Therefore, multiple configuration of tests are conducted and described in this chapter so one can analyze the results from different aspects.  \\

\section{Test with optimal response time}
In this test, fixed variable is average end-to-end response time which is kept under 8 ms  as a criteria for optimal performance. An additional bar is set on the CPU consumption which can not exceed 100\%. \\
In the context of Cloud Foundry, it is impossible to assure an application using only one core. Setting the memory of application can limit the CPU shares it can get. Nevertheless, if other applications on the same node are idling, even the tiniest application can get all the 4 CPUs of the node. In terms of Java application, it is already much more memory consuming than node as shown in \ref{memory} hence it is highly unlikely only one CPU is given to the application. What one witnesses as a 70\% CPU is also very likely distributed in several CPUs.\\
Figure\ref{cpu-100} shows the result of the test described above. It can be deducted from the graph, that Java has a better performance in comparison with Node.js. As the CPU distribution is not unclear,  it can be accounted for that Java application is likely running with 4 or 2 very relaxed CPUs while Node.js is grabbing every bit of computing resource from that single CPU. This also leads to the reflection on the results of benchmarking from TechPower \citep{benchmark}. In terms of applications on cloud, Java is twice more efficient than Node.js. It is possible they haven't taken the CPU distribution into consideration. 

\begin{figure}[h]
	\centering
	\includegraphics[width=12cm]{cpu-100}
	\caption{Compare throughput under 100 CPU consumption}
	\label{cpu-100}
\end{figure}

\section{Test with real load}
\subsection{Test configuration}
 In this round of test, applications are going to scale. Then how should it be contemplated whether two applications are equally scaled? As found out earlier, one Java instance utilizes more CPU while one Node.js employs only one, which leads the test to comparing vertical scaling in Java application with horizontal scaling in Node.js. This is still feasible if one can be sure Java always obtains the same CPUs. However, figure \ref{java-cpu-limit} shows, that Java application never get to use all of its 4 CPUs. The database connection and router load are checked and they are not under stress at all.  An abnormal increase in memory is observed which indicates something suspicious in the application. The implementation is scrutinized, nevertheless, in the thesis no obvious cause is found. This leads to the realization that even though one Java instance gains 4 CPUs, it doesn't equal to four one-CPU Node instances because somehow not all 4 CPUs are put in use. 
 
 \begin{figure}[h]
 	\centering
 	\includegraphics[width=12cm]{java-cpu-limit}
 	\caption{Limited CPU usage in Java application}
 	\label{java-cpu-limit}
 \end{figure}
 
 Therefore, it is decided in this thesis to compare the performance by normalizing the throughput produced by a not overloaded CPU. For example, application achieves a 100000 throughput without a significant increase in response time. Its CPU usage is 250\%. Then the normalization of throughput is 100000/250=400. \\ 
Load is brought about through sending one request on application per instance of load generator.It is decided not to send parallel requests so as to guarantee each instance of the generator is simulating one end user. With each new round of test, 4 instances will be added to the existing running load generator.\\
 For Java application, the increase on load stops when the throughput from application stops growing while the response time climbs up. Also, the load stops when router has reached its limit (70\% load). For Node.js applications, load stops accumulating when the CPU usage has reached over 90\% or equals the maximum CPU usage of Java applications tested.
 
\subsection{Test result}
 \begin{figure}[h]
	\centering
	\includegraphics[width=12cm]{all-app-cpu}
	\caption{Throughput in relation to CPU}
	\label{all-app-cpu}
\end{figure}
In figure \ref{all-app-cpu}, it is illustrated the throughput in relation to the CPU usage of Java application in running instance numbers of 1, 2, and 3. The xx axis is the throughput , the yy axis is the CPU consumption. When 3 Java applications are running, the router reaches its maximum capacity and becomes bottleneck as no more requests can be handled without losing time spent waiting for the load balancing in router. As summarized in table \ref{app-cpu-usage}, with every increase of instance, the CPU usage from total available CPU decreases, which indicates the application doesn't scale horizontally by adding more instances. 
\begin{table}[h]
	\caption{Percentage of utilized CPU}
	\label{app-cpu-usage}
	\renewcommand{\arraystretch}{1.2}
	\centering
	\sffamily
	\begin{footnotesize}
		\begin{tabular}{l l l l l  }
			\toprule
			\textbf{Application  type} &\textbf{Instance Number} & \textbf{Available CPU \%} & \textbf{Maximum CPU\%}& \textbf{Utilzed CPU} }\\
		\midrule
		Java &1 	&	229	 & 400 & 57\% \\
		Java &2	&	361 & 800& 45\% \\
		Java &3	&	429  &	1200 & 35\%\\
				\midrule
			Node.js &1 	&	103	 & 100 & 103\% \\
		Node.js &2	&	198 &  200& 99\% \\
		Node.js &3	&	291 & 300 & 97\%\\
		Node.js &4	&	384 & 400 & 96\%\\
		Node.js &5	&	471 & 500 & 94\%\\
		\bottomrule
	\end{tabular}
\end{footnotesize}
\rmfamily
\end{table}

The same performance information of Node.js application is also depicted in figure \ref{all-app-cpu}. The test ends with 5 instances of Node.js application , which is when the CPU usage is in the same range of Java one. As summarized in table \ref{app-cpu-usage}, application shows a quite steady utilization of the available CPU resources and scale almost horizontally. \\
Figure \ref{all-app-rt} presents average response time of each test round on the yy axis in relation to throughput on xx axis. It helps to determine whether the CPU is under great stress when producing the throughput. One can't decide application runs better on the sole account that it handles more requests. Longer average response time allows application to parallelize more actions, which can hardly be considered a better performing application. 
 \begin{figure}[h]
	\centering
	\includegraphics[width=12cm]{all-app-rt}
	\caption{Response time in relation to throughput}
	\label{all-app-rt}
\end{figure}
As mentioned earlier, a criteria to decides whether the application is working efficiently is to see whether the throughput increases without a surge in response time. In the graph, some critical points are selected according to criteria. The response time before the selected point turns out to be a relatively flat line. After that, it is a clear tendency of increasing. The corresponding usage of CPU is sorted out and listed. We can say at the chosen points, the CPU still works with no compelling stress. Table\ref{app-cpu-normalized} shows a normalization of the throughput on CPU.
\begin{table}[h]
	\caption{Through put in relation to CPU}
	\label{app-cpu-normalized}
	\renewcommand{\arraystretch}{1.2}
	\centering
	\sffamily
	\begin{footnotesize}
		\begin{tabular}{l l l l l  }
			\toprule
			\textbf{Application  type} &\textbf{Instance Number} & \textbf{CPU \%} & \textbf{Throughput}& \textbf{Normalized value} }\\
		\midrule
		Java &1 	&	196	 & 260701 & 1330\\
		Java &2	&	299 & 410399& 1372\\
		Java &3	&	373  &	525698 & 1409\\
		\midrule
		Node.js &1 	&	83	 & 89943 & 1070\\
		Node.js &2	&	173 &  224948&1300\\
		Node.js &3	&	267 & 303474 & 1136\\
		Node.js &4	&	338 & 397007 &1174\\
		Node.js &5	&	421 & 501403 & 1190\\
		\bottomrule
	\end{tabular}
\end{footnotesize}
\rmfamily
\end{table}

It can be deducted from the table that even though Java application scales inefficiently through adding instance, in terms of the CPU it scales almost linear. Comparing with Node.js application, it yields more throughput for every percentage of CPU which leads to the conclusion that it is more CPU efficient. \\
The last metric to check is memory consumption. As displayed clearly in figure \ref{all-app-memory}, Node.js shows excellent memory utilization for the whole test range while Java is quite memory demanding. Although it offers some performance gains in terms of handling more requests than the Node.js, it is not in proportion with the memory consumption. 
 \begin{figure}[h]
	\centering
	\includegraphics[width=12cm]{all-app-mem}
	\caption{Memory in relation to CPU}
	\label{all-app-memory}
\end{figure} % Externe Datei einbinden
\chapter{Conclusion}
The aim of the thesis is to compare the performance of Java and Node.js with the focus on SAP use case in Cloud Foundry. The example applications are of minimum complexity and functionality. The consumption of resources are also kept at a low level. Still, both technologies have reached 20\% of average requests handling of Google.com\citep{Google}. This is a quite good and firm performance for SAP business scenario. The CPU usage would rarely reach 100\% before a limitation in the database or router surfaces first. Of course, the application should be properly optimized first and have an efficient CPU usage.\\
The Node.js application has demonstrated a very high utilization of available CPU and memory efficiency, which also makes it suitable for gaining more performance through horizontal scaling. This also have some implications for developers regarding component design. If the cloud application needs to be horizontally scaled up or down at any moment according to the load condition, the components should be designed to be as stateless as possible.\\
Java exhibits a very good CPU efficiency. If the cloud provider charges the cost according to consumed CPU resources instead of reserved CPU total, Java definitely offers a better performance gain. However, if it is not the case, a deep investigation should be conducted into the CPU utilization of the application. If Java can efficiently consume all available CPU power, it offers a great vertical scalability. Despite of the fact that one Java instance consumes more memory, if it can make good use of 64-core system, it would outperform 64 small Node.js instances in terms of memory and CPU. As for horizontal scaling for an optimized Java application, it will achieve possibly the same computing capability but needs a much higher memory consumption. \\
However, no matter Java or Node.js, the scaling competence goes beyond application itself in context of cloud. In the thesis, we have faced a bottleneck imposed from HA-Proxy before we even scale the application at all. This limitation can only be work around at the application level with reusing established connections. There is nothing one can do at the infrastructure level which would mean to rebuild the system since HA-proxy is not built scalable. In comparison, the scaling of Gorouter is done in one day and has been operating perfectly ever since. There are even more to the scaling in cloud: the VMs or container, the database... Every scaling also puts demand on operation and monitoring. For example, high load could cause a defect in monitoring if it is not built to cope with a sudden large amount of reading and writing data entries. It will trigger false alarms and result in false evaluation of cloud platform health.  % Externe Datei einbinden
\chapter{Future Work}
Scalability of applications is an enormous topic. The thesis has only scratched the surface a little bit. There are several additional future directions that can be explored. In order to eliminate overhead brought by framework, the applications have a minimum utilization of them. It would be interesting to know what kind of influence they might bring about. However, of course, instead of testing the technology itself, the framework competence is tested. Some work has been done by TechEmpower \citep{Benchmark}. They have benchmarked the library and framework overhead for single application. It would be interesting to know if there is any variation in respect to scaling. The testing scenario could also be expanded to computation intensive cases, like machine learning and see how the two technologies cope with calculating. \\
Regarding applications, research can be conducted in whether a cluster module of Node.js application can still demonstrate a high utilization of processing power and the strange phenomenon of Java's inefficient consumption of available computing resource is also worth looking into. Load generating concept in the thesis has fulfilled some requirement that are overseen by the general tools. It would be nice to have a further development of load generator to enrich its functionality and eliminate manual process like monitoring and recording CPU and memory consumption. \\ % Externe Datei einbinden
% ------------------------------------------------------------------

\label{lastpage}

% Neue Seite
\cleardoublepage

% Backmatter mit normalem Zeilenabstand setzen
\singlespacing

% Römische Ziffern für die "Back-Matter", fortlaufend mit "Front-Matter"
\pagenumbering{roman}
\setcounter{page}{\value{frontmatterpage}}

% Abkürzungsverzeichnis
\chapter*{Abbreviations}
\addcontentsline{toc}{chapter}{Abbreviations}

\begin{acronym}
\acro{DEA}{Droplet Execution Agents}
\acro{PaaS}{Platform as a Service}
\acro{AWS}{Amazon Web Service}
\acro{IEEE}{Institute of Electrical and Electronics Engineers}
\acro{ISO}{International Organization for Standardization}
\acro{BBS}{Bulletin Board System}
\acro{ELK}{Elasticsearch, Logstash, and Kibana}
\acro{TCO}{Total Cost of Ownership}
\acro{ASF}{Apache Software Foundation}
\end{acronym}


% Tabellenverzeichnis erzeugen
\cleardoublepage
\phantomsection
\addcontentsline{toc}{chapter}{List of Tables}
\listoftables

% Abbildungsverzeichnis erzeugen
\cleardoublepage
\phantomsection
\addcontentsline{toc}{chapter}{List of Figures}
\listoffigures

% Listingverzeichnis erzeugen
%\cleardoublepage
%\phantomsection
%\addcontentsline{toc}{chapter}{Source references}
%\lstlistoflistings

% Literaturverzeichnis erzeugen (nach DIN)
\begin{flushleft}
\bibliographystyle{dinat} % Literatur nach DIN
%\bibliographystyle{abbrv} % Zitate mit [1], [2] etc.
%\bibliographystyle{alpha} % Zitate mit [Kor01], [Vix99] etc.
\bibliography{literatur}   % BibTeX-Datei mit Literaturquellen einbinden
\end{flushleft}


% Index ausgeben
\cleardoublepage
\phantomsection
\addcontentsline{toc}{chapter}{Index}
\printindex

\listoftodos

\end{document}