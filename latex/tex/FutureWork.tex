\chapter{Future Work}
Scalability of applications is an enormous topic. The thesis has only scratched the surface a little bit. There are several additional future directions that can be explored. In order to eliminate overhead brought by framework, the applications have a minimum utilization of them. It would be interesting to know what kind of influence they might bring about. However, of course, instead of testing the technology itself, the framework competence is tested. Some work has been done by TechEmpower \citep{benchmark}. They have benchmarked the library and framework overhead for single application. It would be interesting to know if there is any variation in respect to scaling. The testing scenario could also be expanded to computation intensive cases, like machine learning and see how the two technologies cope with calculating. \\
Regarding applications, research can be conducted in whether a cluster module of Node.js application can still demonstrate a high utilization of processing power and the strange phenomenon of Java's inefficient consumption of available computing resource is also worth looking into. Load generating concept in the thesis has fulfilled some requirement that are overseen by the general tools. It would be nice to have a further development of load generator to enrich its functionality and eliminate manual process like monitoring and recording CPU and memory consumption. \\