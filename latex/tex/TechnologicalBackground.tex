% Die Arbeit besteht aus Kapiteln (chapter)
\chapter{Technological Background}
Performance and load testing has been conducted ever since applications came into been. There are a sea of different tools available in internet. In the case of this paper, the server's capacity is going to be driven to its limit. After that server won't stay overloaded. Instead, it will try to scale and consume more clients. This requires the testing tool should be capable of generating a large enough amount of end users. 

One of the most popular testing tools is JMeter which is based on Java and cross-platform. It supports multiple protocol and has a very friendly UI to configure test plans. It even produces aggregated test results in different type of graphs. It seems it has everything one asks for except scalibility. When the test requires a bigger scale, JMeter client machine quickly runs in to issue that it is unable, performance-wise, to simulate enough users to stress the server or is limited at network level, an option exists to control multiple, remote JMeter engines from a single JMeter client. It By running JMeter remotely, you can replicate a test across many low-end computers and thus simulate a larger load on the server. One instance of the JMeter client can control any number of remote JMeter instances, and collect all the data from them. This offers the following features:

  
