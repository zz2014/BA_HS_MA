% Die Arbeit besteht aus Kapiteln (chapter)
\chapter{Introduction}
The promise of cloud has ushered the software engineering into a new era. Technology is to be used anytime and anywhere to untangle issues, provide solutions, bring people together and change their ways of living. This new era of net-centric web-based applications redefines how software is delivered to a customer and how the customer uses the delivered software. The digitization of economy impacts everyone. New businesses and leaders are emerging from nowhere. Companies appear and disappear much faster than ever before. 

SAP , a company which has been renown for its on premise ERP solutions, also faces the pressure from customers to reduce TCO and increase agility. It has since long announced its cloud strategy to help customer master the digital economy. The reach of SAP systems is to be extended through cloud based access, ultimately reaching everybody everywhere in entirely new ways. Cloud Computing allows on demand software provisioning with Zero-Installation and automatic configuration at low cost and immediate access in ultra-scalable data centers, which leads to the next generation networked solutions. In the process of leverage cloud infrastructure to increase the business agility and to lower the TCO of customers, ultimately new types of applications are enabled.

Cloud solutions come with new capabilities and technical challenges for cloud software development.  Different cloud software development platforms and frameworks that applied with different programming languages are used inside SAP. JAVA EE and Node.JS are the most prominent ones. There have been on-going discussions about how to  take a choice between them for SAP applications considering the SAP environment. This paper aims to discuss this issue in terms of their respective performance and scalibility. The reason is simple. Among all the major advantages brought by the cloud paradigm, scalability is the one that makes cloud computing different to an "advanced outsourcing" solution. Performance, which heavily affects customer satisfaction is a key metric. 
\todo{Write more about project introduction}


%\section{Zweite Ebene}

% Unterkapitel können noch einmal durch subsections untergliedert 
% werden (jetzt auf der 3. Ebene)
%\subsection{Dritte Ebene}

% Mit Labels können Sie sich später im Text wieder auf diese Stelle beziehen
%\label{Gliederung:EbeneDrei}

% Einträge für den Index anlegen. Ein Index wird normalerweise in einer Abschluss
% Arbeit nicht benötigt.
%\index{Gliederung!Ebenen}

% Auf der 4. Ebene liegen die subsubsections. In diesem Template bekommt die
% 4. Ebene keinen Nummern mehr und erscheint auch nicht im Inhaltsverzeichnis
%\subsubsection{Vierte Ebene}

% Auf der 5. Ebene werden einzelne Absätze mit Überschriften versehen.
%\paragraph{Fünfte Ebene} Anders als in diesem Beispiel, darf in Ihrer Arbeit kein Gliederungspunkt auf seiner Ebene alleine stehen. D.\,h. wenn es ein 1.1 gibt, muss es auch ein 1.2 geben.
