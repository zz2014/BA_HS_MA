% -------------------------------------------------------
% Daten für die Arbeit
% Wenn hier alles korrekt eingetragen wurde, wird das Titelblatt
% automatisch generiert. D.h. die Datei titelblatt.tex muss nicht mehr
% angepasst werden.

\newcommand{\hsmasprache}{en} % de oder en für Deutsch oder Englisch

% Titel der Arbeit auf Deutsch
\newcommand{\hsmatitelde}{Einsatz eines Flux-Kompensators für Zeitreisen mit einer maximalen Höchstgeschwindigkeit von WARP~7}

% Titel der Arbeit auf Englisch
\newcommand{\hsmatitelen}{Performance Comparison between Java and Node.js in terms of Scaling in Cloud}

% Weitere Informationen zur Arbeit
\newcommand{\hsmaort}{Mannheim}    % Ort
\newcommand{\hsmaautorvname}{Zheng} % Vorname(n)
\newcommand{\hsmaautornname}{Zeng} % Nachname(n)
\newcommand{\hsmadatum}{15.02.2017} % Datum der Abgabe
\newcommand{\hsmajahr}{2017} % Jahr der Abgabe
\newcommand{\hsmafirma}{Paukenschlag GmbH, Mannheim} % Firma bei der die Arbeit durchgeführt wurde
\newcommand{\hsmabetreuer}{Prof. Peter Knauber, Hochschule Mannheim} % Betreuer an der Hochschule
\newcommand{\hsmazweitkorrektor}{Jens Keller,  SAP SE} % Betreuer im Unternehmen oder Zweitkorrektor
\newcommand{\hsmafakultaet}{I} % I für Informatik
\newcommand{\hsmastudiengang}{IB} % IB IMB UIB IM MTB

% Zustimmung zur Veröffentlichung
\setboolean{hsmapublizieren}{true}   % Einer Veröffentlichung wird zugestimmt
\setboolean{hsmasperrvermerk}{false} % Die Arbeit hat keinen Sperrvermerk

% -------------------------------------------------------
% Abstract

% Kurze (maximal halbseitige) Beschreibung, worum es in der Arbeit geht auf Deutsch
\newcommand{\hsmaabstractde}{}

% Kurze (maximal halbseitige) Beschreibung, worum es in der Arbeit geht auf Englisch

\newcommand{\hsmaabstracten}{}



